\documentclass[a4paper]{article}
\usepackage[T1]{fontenc}
\usepackage[utf8]{inputenc}
\usepackage[english]{babel}
\usepackage{macros}

\title{SAML: the implementation}
\author{Samuel Mimram}

\begin{document}
\maketitle

\section{Type inference}
\subsection{Declarations}
Declarations can either be of the form
\begin{verbatim}
let x = e in e'
\end{verbatim}
or (sequences of)
\begin{verbatim}
x = e
\end{verbatim}
The first one has the advantage that program transformations are more easily
implemented. For instance, with the second one, the typing function has to
return the environment.

\subsection{Application}
\texttt{f(y)} reduces to
\begin{verbatim}
let x = y in f
\end{verbatim}
where \texttt{x} is the name of the variable of the function, in order to handle
the case where \texttt{y} has effects (in which case it should not be inlined).

\section{The monad}
The ``expansion'' of the monad returns a record with two fields
\begin{verbatim}
alloc : unit -> record
loop : init:bool -> record -> 'a
\end{verbatim}

\end{document}
